\documentclass[11pt]{article}
\usepackage[brazilian]{babel}
\usepackage[utf8]{inputenc} %Deixa eu colocar letras com ascentos
\usepackage[T1]{fontenc}
\usepackage{amsmath}
\usepackage{color}
\usepackage{graphicx}


\title{Relatório - EP Fase 3 \\ Laboratório de Programação 2}



\begin{document}

\maketitle

\section{Integrantes}

\begin{itemize}

\item Victor Sanches Portella - Nº USP: 7991152

\item Lucas Dário - Nº USP: 7990940

\item Ruan Costa - Nº USP: 7990929

\end{itemize}

\section{Introdução}

Nesta fase foram implementados a parte visual do jogo, algumas instruções adicionais para melhor desenvolvimento dos robos através do nosso \emph{Assembly}, e foram impelmentados alguns melhoramentos de algumas instruções. Este relatório tem como objetivo dar uma visão geral de todas essas mudanças.

\section{Interface gráfica}

Toda a parte da interface gráfica está implementada no arquivo \textbf{\color{red}MapaVisual.java}. Neste arquivo há diversas classes e estruturas que controlam a parte gráfica do jogo. Uma breve explicação sobre algumas dessas estruturas:

\begin{itemize}

	\item \textbf{classe Célula}: Representa um hexagono do do campo. Seus atributos são o  centro (x, y), o raio, a imagem dessa célula e o hexagono(polígono). O construtor constrói o hexagono de maneira similar à disponibilizada no paca. A função draw desenha uma célula

	\item \textbf{classe Campo}: Extensão de JPanel. Tem como atributos uma matriz de células, o mapa, um boolean que diz se o jogo acabou e as imagens usadas no jogo. O construtor seta o mapa e cria um campo com hexagonos de lado L. 
	
	A função \textbf{\color{red}criaCampo()} carrega as imagens e varre a matriz de células configurando suas respectivas coordenadas de centro e a imagem de fundo de cada célula, de acordo com o Terreno que aquela célula representa.
	
	A função \textbf{\color{red}paintComponent(Graphics g)} é um override da função de mesmo nome implementada em JPanel. Percorre o campo desenhando os hexagonos com suas respectivas imagens. Caso o jogo já tenha acabado, também exibe uma imagem de game over.
	
	A função \textbf{\color{red}desenhaElemento(...)} desenha o elemento passado em cima da imagem de fundo da célula de coordenadas passadas. É usada para desenhar robos e bases.
	
	A função \textbf{\color{red}getScaledImage(...)} converte a imagem passada por argumento para uma outra com as dimensões passadas também por argumento.

	\item \textbf{classe Tela}: Extensão de JFrame. O construtor dá o título da janela, seta seu tamanho, adiciona um listener e adiciona o JPanel campo.

	\item \textbf{classe MapaVisual}: Classe que interfaceia a comunicação entre a estrutura do jogo e a parte gráfica. Possui como atributos o mapa, os valores estáticos de largura, altura (da janela) e lado do hexagono (em pixels), um objeto Tela e um objeot Campo.	O construtor apenas seta os atributos.
	
	A função \textbf{\color{red}atualiza()} repinta a tela(que é uma janela).
	
	A função \textbf{\color{red}abreJanela()} seta os atributos campo e tela e cria um Runnable, tal que o método sobrecarregado run() apenas faz com que a tela(janela), seja visível.
	
	A função \textbf{\color{red}gameOver()} seta para true o boolean gameOver do atributo campo e assim encerra o jogo.

\end{itemize}


\section{Arquivos de mapas}

Nessa fase, para uma edição e criação facilitada de mapas, fizemos com que os mapas fossem gerados a partir de um aruqivo 
de entrada de texto.
Exemplo:
\begin{verbatim}

6 6
1 1 1 1 1 1
1 -1 0 0 0 1
1 0 1 1 0 1
1 0 1 1 0 1
1 0 0 0 0 1
1 2 2 2 2 1

\end{verbatim}

Na primeira linha deve ter um par de números representando o número de \textbf{linhas} e de \textbf{colunas}, respectivamente.
Em seguida, cada linha de números representa uma linha do mapa, onde cada número representa um terreno. Legenda dos números:

\begin{center}
	\begin{tabular}{ | c | c |}
	\hline
	Número & Terreno\\
	\hline 
	-N & Base do time de Id N\\
	\hline
	0 & Liso\\
	\hline
	1 & Rugoso\\
	\hline
	2 & Água\\
	\hline
	3 & Depósito de cristais\\
	\hline
	Default & Liso \\
	\hline

	\end{tabular} 
\end{center}

\section{Novas instruções}

\begin{itemize}

\item \textbf{\color{red}MOD}: Desempilha dois números da pilha, e empilha o resto da divisão do segundo pelo primeiro.

\item \textbf{\color{red}GETPOS}: Desempilha um Terreno da pilha, e empilha a posição do mesmo (obejto \textbf{Posicao}) do mesmo como resposta. 

\item \textbf{\color{red}GETX}: Desempilha um objeto Posicao da pilha, e empilha sua coordenada x como resposta.

\item \textbf{\color{red}GETY}: Desempilha um objeto Posicao da pilha, e empilha sua coordenada y como resposta.

\item \textbf{\color{red}TIMEID}: Desempilha um time da pilha de dados, e empilha o número que representa aquela time (o ID do time).

\item \textbf{\color{red}GETBASE}: Desempilha um time da pilha de dados, e empilha o Terreno do tipo Base daquele time.

\item \textbf{\color{red}ISA}: Desempilha um número e um empilhável. Esse comando irá empilhar 1 caso o objeto desempilhado seja do tipo representado por aquele número (Vide tabela abaixo), e 0 c.c.
\begin{center}
	\begin{tabular}{ | c | c |}
	\hline
	Número & Tipo\\
	\hline
	0 & Numero\\
	\hline
	1 & Terreno\\
	\hline
	2 & Liso\\
	\hline
	3 & Robo\\
	\hline
	4 & Deposito\\
	\hline
	5 & Rugoso\\
	\hline
	\end{tabular}

\end{center}

\item \textbf{\color{red}GETROBO}: Desempilha um número, representando uma direção, e empilha o Robo que está naquele terreno, e empilha 0 caso não haja Robo. 

\item \textbf{\color{red}GETTIME}: Desempilha um número, e é empilhado como resposta o time correspondente a aquele número. Empilha 0 caso não existe tal time.

\item \textbf{\color{red}MYTIME}: Empilha o time do próprio robo.

\item \textbf{\color{red}LOOK}: Desempilha um número representando uma direção, e empilha o terreno daquela posição. Empilha 0 caso não exista nada naquela direção.

\end{itemize}

Além disso, os comandos de chamada ao sistema: \textbf{WALK}, \textbf{ATK}, \textbf{COLLECT} e \textbf{DROP} usam como argumento de direção um número no topo da pilha da máquina virtual.
Outra modificação foi que, para evitar que o robô se "teletransportasse", já que a cada iteração cada robo executa 5 comandos de máquina, fizemos com que o robô acabe sua
rodada ao tentar se mover. Caso isso não acontecesse, seria possível que ele conseguisse andar duas vezes em uma só iteração, dando a impressão que o robô pudesse "voar".






\section{Tradutor em Perl}

Nessa fase, o nosso tradutor em Perl recebe o mapa, diversos robos como argumento e é possível separá-los em diversos times. Apesar do número de times ser indefinido,
nossa interface gráfica suporta somente 2 times. O mapa deve ser estar na pasta \textbf{\color{red}data/map}. Os robos na pasta \textbf{\color{red}test} e com extensão .asm.
Sintaxe:

\begin{verbatim}

$>perl bin/montador.pl mapa3.txt 1 robo1_time1 robo2_time1 2 robo1_time2
  robo2_time2  ... n robo1_timen

\end{verbatim}
Com isso, será gerado um novo arquivo Main.java na pasta src.

\section{Modo de compilação}

Para compilar, execute o seguinte comando:

\begin{verbatim}

$> ant

\end{verbatim}

Com isso, o jogo será compilado, e será gerado um arquivo .jar localizado na pasta dist/. Assim, parara executar o jogo, a sintaxe é:

\begin{verbatim}

$> java -jar dist/EP3.jar

\end{verbatim}






\end{document}

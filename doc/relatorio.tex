\documentclass[11pt]{article}
\usepackage[brazilian]{babel}
\usepackage[utf8]{inputenc} %Deixa eu colocar letras com ascentos
\usepackage[T1]{fontenc}
\usepackage{amsmath}
\usepackage{color}
\usepackage{graphicx}


\title{Relatório - EP Fase 4 \\ Laboratório de Programação 2}



\begin{document}

\maketitle

\section{Integrantes}

\begin{itemize}

\item Victor Sanches Portella - Nº USP: 7991152

\item Lucas Dário - Nº USP: 7990940

\item Ruan Costa - Nº USP: 7990929

\end{itemize}

\section{Introdução}

Nesta fase foram implementados o compilador da linguagem de alto nível do jogo, algumas instruções adicionais para melhor desenvolvimento dos robos através do nosso \emph{Assembly}, e foram impelmentados alguns melhoramentos de algumas instruções. Este relatório tem como objetivo dar uma visão geral de todas essas mudanças.

\section{Compilador}

Optamos por adequar o código do professor ao nosso assembly. Em alguns casos foi necessário incluir instruções novas ao nosso assembly, como é o caso da instrução LRCL, LSTO e outras. O compilador está no arquivo \emph{src/Parser.jj}.
Na linguagem de alto nível, além das funcionalidades que já haviam no compilador do professor, também temos os seguintes métodos:

OBS: Todas as funções abaixo retornam zero, caso os argumentos não sejam do tipo certo.

\begin{itemize}

	\item \textbf{move(int)}: Move o robo para a direção x, caso x esteja entre 1 e 6 e a Arena permita o movimento do robô.

	\item \textbf{look(int)}: Retorna o terreno da direção x, caso x esteja entre 1 e 6 e a Arena permita a ação.
    
    \item \textbf{getRobo(int)}: Retorna o robo da direção dada, caso exista um robô lá.
    
    \item \textbf{look(int)}: Retorna o terreno da direção x, caso x esteja entre 1 e 6 e a Arena permita a ação.
    
    \item \textbf{myTime()}: Retorna o objeto time do robo que está executando a ação.
    
    \item \textbf{atk(int)}: Executa um ataque na direção recebida. Caso haja um robô nessa direção, o mesmo perde 10 de vida(por padrão).
    
    \item \textbf{drop(int)}: Caso o robo esteja carregando um cristal, abandona-o na direção dada, caso essa seja a base inimiga. Se não for, o cristal volta para seu depósito de origem. 
    
    \item \textbf{collect(int)}: Pega o cristal da direção dada, caso haja cristal no terreno daquela direção. Lembre-se que cada robô só pode carregar um cristal.
    
    \item \textbf{getTime(Robo/int)}: Retorna um objeto time. O argmento pode ser tanto um robô quanto um inteiro. No segundo caso, retornará o time cujo ID é o inteiro.
    
    \item \textbf{timeId(Time)}: Retorna o ID do objeto time passado.
    
    \item \textbf{timeRb(Robo)}: Retorna o objeto time do robô passado.
    
    \item \textbf{getX(Posicao)}: Retorna a coordenada x de um objeto posição.
	
    \item \textbf{getY(Posicao)}: Retorna a coordenada y de um objeto posição.
    
    \item \textbf{getPos(Terreno)}: Retorna o objeto Posicao do terreno passado.
    
    \item \textbf{getBase(Time)}: Retorna a base do time passado.
    
    \item \textbf{dropBomba()}: Coloca uma bomba na posição atual do robô. Esta explodirá daqui a 5 rodadas. TODOS ao redor tomam 40 de dano, portanto cuidado.
    
    \item \textbf{x isa y}: Retorna 1 caso x seja do tipo y. y tem que ser uma das palavras reservadas declaradas no final do compilador. Ex: x isa ROBO, x isa BASE, etc..
    
    \item \textbf{x mod y}: Retorna o resto da divisão de x por y.
\end{itemize}


\section{Interface Programável}

Nessa fase começamos a usar a interface Programável. Ela define quais objetos podem ter comportamento programável.


\section{Compilação e execução}

Compilação e execução deve ser feitas dentro da pasta principal, isto é, pasta que contém as pastas bin, dist, etc...
Para compilar, execute o seguinte comando:


\begin{verbatim}

$> make

\end{verbatim}
OBS: supomos que o javacc está instalado.
\\Com isso, o jogo será compilado, e será gerado um arquivo .jar localizado na pasta dist/. Assim, parara executar o jogo, a sintaxe é:

\begin{verbatim}

$> java -jar dist/EP4.jar 1 robo1 robo2 ... roboN 2 robo1 robo2 ... roboM

\end{verbatim}

Por exemplo: 

\begin{verbatim}
java -jar dist/EP4.jar 1 roboEzio 2 roboAndador
\end{verbatim}






\end{document}
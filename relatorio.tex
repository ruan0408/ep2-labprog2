\documentclass[11pt]{article}
\usepackage[brazilian]{babel}
\usepackage[utf8]{inputenc} %Deixa eu colocar letras com ascentos
\usepackage[T1]{fontenc}
\usepackage{amsmath}
\usepackage{color}


\title{Relatório - EP Fase 2 \\ Laboratório de Programação 2}



\begin{document}

\maketitle

\section{Integrantes}

\begin{itemize}

\item Victor Sanches Portella - Nº USP: 7991152

\item Lucas Dário - Nº USP: 7990940

\item Ruan Costa - Nº USP: 7990929

\end{itemize}

\section{Introdução}

Neste relatório temos como objetivo dar uma explicação concisa sobre a organização das classes no funcionamento do programa atualmente. Para isso, explicaremos o funcionamento do progrmaa inteiro, e iremos nos aprofundar nos assuntos que forem necessários.


\section{Funcionamento}

Nessa fase o EP ainda não têm o funcionamento de um jogo em si, mas já é possível ver diversos robos agindo em paralelo. Para isso, o script em Perl\footnote{Explicação do mesmo feita mais a frente} traduz diversos programas para um arquivo chamado \textbf{\color{red}Main.java}. 

Todos os robôs traduzidos serão colocados no mesmo time. Ao rodar o programa, os programas de cada robô será executado até o fim. Quando todos os robos tiverem acabado de executar, o programa será finalizado.

Cada robô imprime as instruções que execultou para que seja possível que nós possamos acompanhar a execução de cada robô.

\section{Classes}

\subsection{Classe Arena}


\subsection{Classe Programa}

\section{Interfaces}

\section{Tradutor em Perl}

\end{document}

\documentclass[11pt]{article}
\usepackage[brazilian]{babel}
\usepackage[utf8]{inputenc} %Deixa eu colocar letras com ascentos
\usepackage[T1]{fontenc}
\usepackage{amsmath}
\usepackage{color}


\title{Relatório - EP Fase 2 \\ Laboratório de Programação 2}



\begin{document}

\maketitle

\section{Integrantes}

\begin{itemize}

\item Victor Sanches Portella - Nº USP: 7991152

\item Lucas Dário - Nº USP: 7990940

\item Ruan Costa - Nº USP: 7990929

\end{itemize}

\section{Introdução}

Neste relatório temos como objetivo dar uma explicação concisa sobre a organização das classes no funcionamento do programa atualmente. Para isso, explicaremos o funcionamento do progrmaa inteiro, e iremos nos aprofundar nos assuntos que forem necessários.


\section{Funcionamento}

Nessa fase o EP ainda não têm o funcionamento de um jogo em si, mas já é possível ver diversos robos agindo em paralelo. Para isso, o script em Perl\footnote{Explicação do mesmo feita mais a frente} traduz diversos programas para um arquivo chamado \textbf{\color{red}Main.java}. 

Todos os robôs traduzidos serão colocados no mesmo time. Ao rodar o programa, os programas de cada robô será executado até o fim. Quando todos os robos tiverem acabado de executar, o programa será finalizado.

Cada robô imprime as instruções que execultou para que seja possível que nós possamos acompanhar a execução de cada robô.

\section{Classes}

\subsection{Classe Arena}

\subsection{Classe Maquina}

Um objeto da classe Máquina possui os seguintes atributos:
\begin{itemize}

\item dados
	Objeto do tipo Pilha. É a pilha de dados da maquina.
\item mem
	Vetor de Empilhaveis usado para o robo armazenar constantes.
\item prog
	Objeto do tipo Programa. É o vetor de comandos que aquela máquina executará.
\item vars
	Objeto do tipo variáveis. Guarda o estado atual das váriaveis de determinada máquina.
\item arena
	Referência para a Arena para a realização de chamadas ao sistema.
\item index
	Inteiro que guarda o índice do próximo comando do programa a ser executado.
\item obj
	Objeto do tipo Programável. À princípio é apenas uma referência para o robô que contém aquela máquina.
\end{itemize}

Construtor: Recebe o sistema (arena) com o qual a máquina vai se comunicar e o Programável que a possuíra.
			Mais uma vez, nesse ep, Programaveis são sempre robos. Inicializa mem com tamanho 10 e index setado em 0(zero).
Funções mais importantes:
	void executaProx(): executa o comando cmd da posição index do programa. Para isso chama a função executaCmd(cmd), caso o programa já não tenha chegado ao fim.
	
	int executaCmd(Comando cmd): Executa o comando passado como argumento. Verifica  qual o código do comando e realiza a devida ação.
								 Retorna o índice do pŕoximo comando a ser lido. O sistema é chamado por meio da função sistema(), caso o comando passado precise de permissão da arena(i.e, se for ATK, COLLECT, DROP ou WALK).

	void sistema(Comando cmd):	 Pede para o sistema executar o Comando cmd.

	As outras funções são, em geral, auxíliares e estão bem explicadas nos comentários.

\subsection{Classe Programa}

\section{Interfaces}

\section{Tradutor em Perl}

\end{document}
